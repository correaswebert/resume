%-------------------------
% Resume in Latex
% Author : Sourabh Bajaj
% Website: https://github.com/sb2nov/resume
% License : MIT
%------------------------

\documentclass[letterpaper,11pt]{article}

\usepackage{latexsym}
\usepackage[empty]{fullpage}
\usepackage{titlesec}
\usepackage{marvosym}
\usepackage[usenames,dvipsnames]{color}
\usepackage{verbatim}
\usepackage{enumitem}
\usepackage[pdftex]{hyperref}
\usepackage{fancyhdr}
\usepackage{fontawesome5}
\usepackage{hyperref}
\usepackage{ragged2e}
\hypersetup{
    colorlinks=true,
    linkcolor=blue,
    urlcolor=blue,
    citecolor=blue,
    filecolor=blue
}

\makeatletter
\newcommand{\github}[1]{%
   \href{#1}{\faGithub}%
}
\makeatother

\pagestyle{fancy}
\fancyhf{} % clear all header and footer fields
\fancyfoot{}
\renewcommand{\headrulewidth}{0pt}
\renewcommand{\footrulewidth}{0pt}

% Adjust margins
\addtolength{\oddsidemargin}{-0.525in}
\addtolength{\evensidemargin}{-0.175in}
\addtolength{\textwidth}{1in}
\addtolength{\topmargin}{-.5in}
\addtolength{\textheight}{1.0in}

% \urlstyle{same}

\raggedbottom
\raggedright
\setlength{\tabcolsep}{0in}
\setlength{\footskip}{4.08003pt}

% Sections formatting
\titleformat{\section}
{\vspace{-5pt}\scshape\raggedright\Large}{}{0pt}{}
[\color{black}\titlerule \vspace{-5pt}]

%-------------------------
% Custom commands

\newcommand{\subsectionHeading}[4]{
    % \vspace{-5pt}
    \begin{itemize}[leftmargin=-10pt]
    \item[]
        \begin{tabular*}{\textwidth}{l@{\extracolsep{\fill}}r}
            #1 & #2 \\
            #3 & #4 \\
        \end{tabular*}
    \end{itemize}
    \vspace{-10pt}
}

\newcommand{\subsectionListStart}{
    \begin{itemize}[leftmargin=*]
}
\newcommand{\subsectionListEnd}{
    \end{itemize}
    \vspace{-5pt}
}

\newcommand{\titlePoint}[2]{
    \item{
        \begin{justify}
            \textbf{#1}{: #2 \vspace{-1.5pt}}
        \end{justify}
    }
}

\newcommand{\point}[1]{
    \item{#1} \vspace{-1.5pt}
}

% college, date, degree, extra, gpa
\newcommand{\education}[5]{
    \subsectionHeading
    {\textbf{#1}}
    {#2}
    {#3; #4}
    {#5}
    \vspace{-3pt}
}

\newcommand{\courses}[1]{
    \vspace{5pt}
    \begin{itemize}[leftmargin=-10pt]
    \item[]
        \textbf{Coursework}: #1
    \end{itemize}
    \vspace{-10pt}
}

% title, skills
\newcommand{\skill}[2]{
    \begin{itemize}[leftmargin=-10pt]
    \item[]
        \textbf{#1}: #2
    \end{itemize}
    \vspace{-12pt}
}

% company, place, role, tech stack, date
\newcommand{\experience}[5]{
    \subsectionHeading
    {\uppercase{\textbf{#1}}}
    {#2}
    {#3 $\mid$ #4}
    {#5}
}

\newcommand{\experienceStart}{
    \begin{itemize}[leftmargin=0pt]
}
\newcommand{\experienceEnd}{
    \end{itemize}
    \vspace{-10pt}
}

% project, prof google scholar, prof name, place, role, tech stack, date
\newcommand{\research}[7]{
    \subsectionHeading
    {\textbf{#1} \textit{with} \href{#2}{#3}}
    {#4}
    {#5 $\mid$ #6}
    {#7}
}

\newcommand{\researchStart}{
    \begin{itemize}[leftmargin=0pt]
}
\newcommand{\researchEnd}{
    \end{itemize}
    \vspace{-10pt}
}

\newcommand{\project}[4]{
    \subsectionHeading
    {\textbf{#1} $\mid$ #3 \hspace{1pt} \github{#2} }
    {#4}
    {}
    {}
    \vspace{-15pt}
}

\newcommand{\projectStart}{
    \begin{itemize}[leftmargin=0pt]
}
\newcommand{\projectEnd}{
    \end{itemize}
    \vspace{-10pt}
}

\renewcommand{\labelitemii}{$\circ$}

%-------------------------------------------

\begin{document}

\begin{center}

{\LARGE \uppercase{\textbf{Swebert Correa}}}

\vspace{3pt}

\href{https://www.linkedin.com/in/correaswebert}{linkedin.com/in/correaswebert}
$\cdot$
\href{mailto:correaswebert@gatech.edu}{correaswebert@gatech.edu}
$\cdot$
\href{tel:+14706851680}{(470) 685-1680}
$\cdot$
\href{https://github.com/correaswebert}{github.com/correaswebert}

\end{center}

\vspace{-10pt}

% ---

\section{Education}

\subsectionListStart
    \education
    {Georgia Institute of Technology, Atlanta}
    {Aug 2024 - May 2026}
    {Master of Science in Computer Science}
    {Graduate Teaching Assistant}
    {GPA: 4.0/4.0}

    \education
    {College of Engineering, Pune}
    {Aug 2018 - Oct 2022}
    {Bachelor of Technology in Computer Engineering}
    {Minor in Financial Engineering} 
    {GPA: 9.05/10.0}

    \courses
    {Advanced Operating Systems, Database Implementation, Distributed Computing, Networks}
\subsectionListEnd

% ---

\section{Skills}

\subsectionListStart
    \skill
    {Languages}
    {C, C++, Python, CUDA, Java, Kotlin, Go (Golang), JavaScript, Bash, SQL}
    
    \skill
    {Technologies}
    {Kubernetes, Docker, PostgreSQL, DynamoDB, Coral, GDB, Strace, Nsight Git, Jenkins}
\subsectionListEnd

% ---

\section{Experience}

\subsectionListStart
    
    \experience
    {Amazon}
    {Seattle, WA}
    {Software Development Engineer Intern}
    {Kotlin, Dagger, Coral, Fargate}
    {May 2025 - Aug 2025}

    \experienceStart
        \point
        % {Engineered a real-time service ($\sim 150$ TPS) for Alexa+ that enabled AI-guided device setup by integrating adaptive push/pull delivery logic, improving render rate from 3.84\% to over 50\%}
        {Engineered a real-time notification service serving 150+ TPS for Alexa+ that enabled \href{https://www.aboutamazon.com/news/devices/new-alexa-plus-amazon-devices}{AI-guided device setup}}

        \point
        {Integrated \textbf{adaptive push/pull notification delivery} logic boosting render rate from 3.84\% to over 50\%}
        
        \point
        % {Reduced overhead for sending different notification types by enabling services to integrate with a single microservice instead of more than five, saving about two weeks of onboarding time}
        {Saved 2 weeks ($5\times$ reduction in time) for each on-boarded service to send push, in-app and voice notifications}
    \experienceEnd

    \vspace{-5pt}
    
    \experience
    {Rakuten (Robin.io)}
    {Pune, India}
    {Software Engineer}
    {C, Python, Kubernetes, PostgreSQL}
    {Jun 2022 - Jul 2024}
    \experienceStart      
        \point
        {Built an \href{https://cloud.rakuten.com/blogs/a-next-generation-approach-to-object-storage-edge-ai-cost-and-control}{in-house object storage solution} delivering $\sim 70\%$ feature parity with AWS S3}
        
        \point
        {Designed \textbf{disk rebuild} subsystem for data recovery and built \textbf{disk-sets} to scan 4PB across 250+ disks}
        
        \point
        {Developed asynchronous event-based HTTP \textbf{streaming processor} for \textbf{erasure coding} in API gateway}
        
        \point
        {Implemented \textbf{server-side encryption} using the customer-provided keys across active-active replicated sites}

        \point
        {Mentored an intern in creating a Grafana dashboard for real-time cluster health monitoring}
    \experienceEnd

\subsectionListEnd

% ---

\section{Research Experience}

\subsectionListStart

    \research
    {Nexus Supercomputer Project}
    {https://scholar.google.com/citations?user=FVWrKb0AAAAJ}
    {Prof. Suresh Marru}
    {Atlanta, GA}
    {Research Assistant}
    {C/C++, CUDA, Linux Kernel, Flamegraph}
    {Aug 2025 - May 2026}

    \researchStart
        \point
        {Designing a high-performance distributed file system for the \href{https://news.gatech.edu/news/2025/07/15/georgia-tech-build-20m-national-ai-supercomputer}{supercomputer} in collaboration with the \href{https://www.ncsa.illinois.edu}{NCSA}}
        
        \point
        {Working on FUSE optimizations using passthrough and io\_uring for attachable object storage backends}
        
        \point
        {Exploring RDMA, InfiniBand, and GPUDirect for zero-copy data transfer serving 10PB of NVMe storage}
    \researchEnd
    
\subsectionListEnd

% ---

\section{Projects}

\subsectionListStart

    % \project
    % {Memory Allocator}
    % {https://github.com/correaswebert/malloc}
    % {Operating Systems, C}
    % {Georgia Tech, Fall '25}
    % \projectStart
    %     \point
    %     {Developed a thread-safe malloc family of functions replacement using sbrk and mmap system calls}

    %     \point
    %     {Implemented segregated free-list and free space management algorithms (best-fit, worst-fit and first-fit)}
    % \projectEnd

    % \vspace{-5pt}

    \project
    {Xen Hypervisor Credit Scheduler}
    {https://github.com/correaswebert/malloc}
    {Operating Systems, C}
    {Georgia Tech, Fall '25}
    \projectStart
        \point
        {Implemented type-2 hosted hypervisor leveraging Linux signals for context switching and thread-local storage}
        \point
        {Built M:N user-level threading library with a credit-based scheduler supporting load balancing across vCPUs}
    \projectEnd
    

    \vspace{-5pt}

    \project
    {Deadlock Detection and Resolution}
    {https://buzzdb-docs.readthedocs.io/part2/lab3.html}
    {Databases, C++, Valgrind}
    {Georgia Tech, Spring '25}
    \projectStart
        \point
        {Implemented two-phase locking mechanism for BuzzDB to ensure transaction isolation and atomicity}

        \point
        {Designed deadlock detection using wait-for graph and resolved it based on transaction age and starvation}
    \projectEnd

    \vspace{-5pt}

    \project
    {MIT xv6 Ext2 File System}
    {https://gitlab.com/correaswebert/xv6-ext2}
    {Operating Systems, C, Qemu}
    {COEP, Spring '21}
    \projectStart
        \point
        {Extended xv6 with Ext2 file system, implementing buffer cache management, tree-based searching and logging}

        \point
        {Built a Virtual File System layer with additional file-system APIs and required corresponding syscalls}
    \projectEnd

\subsectionListEnd

\end{document}
